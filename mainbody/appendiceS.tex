
\vspace*{0pt} %【该句涉及到格式设置,请勿乱动】

\appendixsec{支撑材料列表}

% 以下是支撑材料列表的示例
\begin{itemize}[parsep=0pt, itemsep=0pt]\ttfamily
  \item{database.xlsx}
  \item{dataset.csv}
  \item{algorithm\_a.cpp}
  \item{algorithm\_b.m}
  \item{algorithm\_c.py}
\end{itemize}

% windows系统下的电脑有个导出技巧:
%% 1. 来到当前文件夹页面
%% 2. 全选中所有文件
%% 3. 点击文件夹页面中“主页”选项卡的“复制路径”
%% 4. 在空白处粘贴
%% 5. 利用全局替换的方式将根目录统一替换成\item{,然后再单独加上右花括号
%% 6. 做最后编辑(尤其是下划线要加上反斜杠,如上述示例所示),即可得到itemize环境的内容

\appendixsec{相关定理的证明}

\zhlipsum[2]

\appendixsec{程序代码}

%以文件形式插入代码
\lstinputlisting[ language=C++, title={\raggedright\normalsize 计算$\bk{n}$的阶乘(C++):} ]{codes/funfactorial.cpp} %C++

\lstinputlisting[ language=Java, title={\raggedright\normalsize 计算$\bk{n}$的阶乘(Java):} ]{codes/funfactorial.java} %Java

\lstinputlisting[ language=Python, title={\raggedright\normalsize 计算$\bk{n}$的阶乘(Python):} ]{codes/funfactorial.py} %Python

\lstinputlisting[style=Matlab-editor, title={\raggedright\normalsize 计算$\bk{n}$的阶乘(MATLAB):}]{codes/funfactorial.m} %MATLAB(如果不想要这种风格,则把该行命令的可选参数style=Matlab-editor改为language=Matlab




